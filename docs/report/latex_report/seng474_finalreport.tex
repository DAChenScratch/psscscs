\documentclass{article}

% if you need to pass options to natbib, use, e.g.:
%     \PassOptionsToPackage{numbers, compress}{natbib}
% before loading neurips_2019

% ready for submission
% \usepackage{neurips_2019}

% to compile a preprint version, e.g., for submission to arXiv, add add the
% [preprint] option:
%     \usepackage[preprint]{neurips_2019}

% to compile a camera-ready version, add the [final] option, e.g.:
     \usepackage[final]{neurips_2019}

% to avoid loading the natbib package, add option nonatbib:
%     \usepackage[nonatbib]{neurips_2019}

\usepackage[utf8]{inputenc} % allow utf-8 input
\usepackage[T1]{fontenc}    % use 8-bit T1 fonts
\usepackage{hyperref}       % hyperlinks
\usepackage{url}            % simple URL typesetting
% \usepackage{booktabs}       % professional-quality tables
\usepackage{amsfonts}       % blackboard math symbols
% \usepackage{nicefrac}       % compact symbols for 1/2, etc.
% \usepackage{microtype}      % microtypography
% Others added after
\usepackage{breqn}          % break equations over multiple lines
% \usepackage{cite}           % citing references
% \usepackage{biblatex}
\def\BibTeX{{\rm B\kern-.05em{\sc i\kern-.025em b}\kern-.08em
    T\kern-.1667em\lower.7ex\hbox{E}\kern-.125emX}}

% \usepackage[
%   backend=biber,
%   style=alphabetic,
%   sorting=ynt
%   ]{biblatex}
%   \addbibresource{sample.bib}

\usepackage{amsmath}        % for text in math equations


\title{Multiagent Reinforcement Learning in a Synchronous Strategy Game}

\author{%
  Ahmed Siddiqui \\
  Faculty of Computer Science\\
  University of Victoria\\
  Victoria, CA V8P 5C2 \\
  \texttt{jesuisahmedn@gmail.com } \\
  \And
  Louis Kraak  \\
  Faculty of Engineering \\
  University of Victoria \\
  Victoria, CA V8P 5C2 \\
  \texttt{vlkraak@gmail.com} \\
  \And
  Michail Roesli  \\
  Faculty of Engineering \\
  University of Victoria \\
  Victoria, CA V8P 5C2 \\
  \texttt{mroesli@uvic.ca} \\
  \And
  Swapnil Daxini  \\
  Faculty of Physics and Astronomy \\
  University of Victoria \\
  Victoria, CA V8P 5C2 \\
  \texttt{swapneel.dx8@gmail.com} \\
  \And
  Yang Li  \\
  Faculty of Computer Science \\
  University of Victoria \\
  Victoria, CA V8P 5C2 \\
  \texttt{leo19970804@gmail.com } \\
}

\DeclareMathOperator*{\argmax}{arg\,max}
\DeclareMathOperator*{\argmin}{arg\,min}

\begin{document}

\maketitle

\begin{abstract}
  A lot of studies have been done on reinforcement learning recently. Q
  learning, or DQN tries to solve the single agent vs environment problem, where
  some other approaches such as AlphaGo attempt the double agent game. In this
  project, we try to find an algorithm to generate an agent that performs well
  in a multiagent synchronous strategy game.
\end{abstract}

\section{Introduction}

Battlesnake is a synchronous strategy game. It is simply a multiplayer version
of the classic game Snake. The version we are training with is a 11x11 game
board with 4 snakes. In each round, each player chooses an action from the space
\textit{left,right,up,down}. The snake’s head and body will move accordingly. If
a snake’s head runs into any obstacle (a snake’s body or wall), it dies. If it
runs into another snake’s head, the shorter snake dies (in a tied situation both
snakes die). If it runs into a food, its length will grow by 1, and its health
point resets to 100. Every turn, all snake’s health will decrease by 1, and once
it reaches 0 the snake dies. The last snake alive becomes the winner of the
game. If no snakes live, it is considered a draw. See the reference for a more
detailed description \cite{BattlesnakeDoc}.

\section{Related Work}

The AlphaSnake Zero uses the same approach that AlphaGo Zero used
\cite{AlphaGoZeroPaper}, where the training algorithm consists of 3 stages,
self-play, training and pit. Because the lack of a good computing device (a
TPU), the network for AlphaSnake is simplified. AlphaGo uses a two headed
network to compute the value (roughly interpreted as win rate) of the current
game board, and also a policy to guide the next action. As this is a
reinforcement learning problem, there is no true label we can assign to the
training data. Instead, they used a Monte Carlo Tree Search to get a good
estimate of the optimal policy.

\section{Method}

In each iteration, a number of games are played using the current neural
network, the data is recorded for training. For each turn of the game, we keep
the information of the game board (the state), the move each snake made (the
action), the computed Q values, the transition probability (explained later).
Then we use the data to train the network. After training, the new network is
going to compete with the old one and will replace the old one if it wins more.

% Figure Here

\begin{figure}
  \centering
  \fbox{\rule[-.5cm]{0cm}{4cm} \rule[-.5cm]{4cm}{0cm}}
  \caption{Sample figure caption.}
\end{figure}

For the input of the network, we need something that can recover enough
information of the game board. The designed state representation is as the
following. The state is a 21x21x3 tensor. The game board is only 11x11, but it
quickly become a problem that to realize where your head is. To make the task
easier, the view of the game is centered at the snake’s head. Imagine you are
the snake and you view the world around you, so a 21x21 board is needed to
recover all the original board when the snake is at the edge. Every snake gets
its own view of the board.

There are essentially three fundamental things in the game, things that if you
run into you die (body and wall), things that if you run into you grow (food),
and things that if you run into some snake will die (head). Thus, the game board
is classified into 3 channels. The first layer is the picture of all the snakes’
head. The second one is for all the obstacles (body and wall). And the last one
shows all the food on the board. In all layers, 0 means an empty cell.

In the first layer, the value of a snake’s head is the snake’s length relative
to the current player, so a negative value means a shorter snake’s head and vice
versa. Since a snake of the same length as you also kills you, the value of its
head is also negative. Overall, the value roughly tells you how dangerous a
block is. The formula for computing the head value is simply

\begin{equation}
  value = (\text{length of the snake}-\text{length of you}+0.5) \times \text{multiplier}
\end{equation}

where the multiplier is to keep the value in the range of [-1, 1] for training
purposes. The 0.5 solves two problems, one mentioned earlier about the snake of
same length, and it avoids the value inside the brackets to be zero, causing an
“invisible head”.

In the second layer, the wall is simply represented with 1. We also need
something to recover the direction of a body bock, since it holds the
information about how the snake moves on the board. But what is really essential
is how long a body block will remain at its position. If we represent the body
as the number of turns until it will be away, then we can roughly recover the
information of a snake by tracing the increasing value of the body blocks. The
number of turns until it will be removed is simply the body length measured from
that body block to the tail of the snake. The value is calculated using the
following formula

\begin{equation}
  value = (\text{number of turns until it's removed}) \times \text{multiplier}
\end{equation}

There are degenerate cases where it is not possible to tell which body belongs
to which snake, but body blocks all work the same way, so it does not matter.

In the last layer, all the food is labeled according to your health point. This
has a huge benefit to reducing the difficulty of learning. It strongly relates
the food to health, and no explicit information for health is needed to be
stored.

\begin{equation}
  value = (101-\text{your health}) \times \text{multiplier}
\end{equation}

Notice that it is impossible to recover the health of other snakes, but
according to the experience of Battlesnake tournaments, it is not a useful
information, as death due to starvation almost never happens in high level
games. We use 101 instead of 100 to avoid the value inside the brackets to be
zero, resulting in “invisible food”.

For the output of the network, we need something that can guide the action to
take. A policy could be a good choice, but unlike the game of Go, where a move
usually just adds another piece on the board, MCTS is too expensive for
Battlesnake, because a move will drastically change the game board and thus very
slow to compute. Instead of approximating the optimal policy, it is easy to
obtain a good estimation of the Q values. Assume the optimal function for Q
value is $Q(s,a)$, where it takes as input the current state $s$ and the action
$a$, and it outputs the value (win rate) of the player if that action $a$ is
chosen under state $s$. An obvious fact is that a game board of Battlesnake is
equivalent to its rotation and reflections (with actions also rotated or
reflected, of course). Thus, we can rotate the board so that the snake is always
facing upwards. This mainly has two benefits. First, it automatically treats all
rotations as one, thus reduce overfitting. Second, there is always a backward
move immediately kills you since you run into your own body, so this action is
meaningless to evaluate. If the snake is always facing up, the output space can
be transformed into \textit{turn left,go straight,turn right}, represented by
numeric values {0,1,2}. It simplifies the network and training. Although it is
true that only for the very first move, there are technically four choices, but
we don’t care about that one move so much that we want to expand the output
space. During the training, we can also make the reflections of the board and Q
value vectors to force the network to realize symmetry, and hopefully reduce
overfitting. Once we have the 3 dimensional vector $\langle Q(s,0),Q(s,1),Q(s,2)
  \rangle$, namely the Q value of each action $a_i$ given the current state s, a
softmax-like function is going to generate a probability mass function based on
the values, which is the policy. The softmax-like function, denoted $\pi$, is simply
softmax with exponent n instead of e.

\begin{equation}
  \pi_n(\overrightarrow{v})=\overrightarrow{v}^{\circ n}\cdot(\sum\limits_i \overrightarrow{v}^{\circ n}i)^{-1}
\end{equation}

Here, $\overrightarrow{v}$ is a vector and $n$ is a constant.
$\overrightarrow{v}^{\circ n}$ denotes the element-wise power. By picking
the n, we can control how explorative the snake is, the higher n is, the
more it highlights the higher values in $\overrightarrow{v}$, and thus it
would not bother to pick the lower valued actions.

To get a good estimate of the value function without MCTS, we label the Q values
according to the following criteria. From the point of view of any snake, the
transition between states is not deterministic. The transition is a Markov
process because it depends on how other snake moves. Let the function
$\Phi(s_0,a,s_1)$ represents the probability of entering state $s_1$, if you
take the action a at current state $s_0$. Let the Q value vector of state $s_0$
be $\langle Q(s_0,0),Q(s_0,1),Q(s_0,2) \rangle$. Say $Q(s_0,a)$ is high and you
pick this action a. Now with probability $\Phi(s_0,a,s_1)$, we then have entered
the next state $s_1$, and we now get the Q values $\langle
  Q(s_1,0),Q(s_1,1),Q(s_1,2) \rangle$. Since the Q values represent win rate, the
optimal strategy is to pick $\argmax_{a_i}Q(s_1,a_i)$. Then we can say that, at
state $s_0$, if we pick action a, with probability $\Phi(s_0,a,s_1)$, we end up
with a win rate of $\max_{a_i}Q(s_1,a_i)$. Obviously, we can see that the Q
value is a weighted sum of the max Q values of its successor states, thus we
have

\begin{equation}
  Q(s_0, a) = \sum\limits_{s_i}\Phi(s_0,a,s_i)\max_{a_j} Q(s_i,a_j)
\end{equation}

This equation is the core part of AlphaSnake Zero’s training algorithm. We can
now refine the Q value vector by updating its explored entry $Q(s_0,a)$. A
complete record of a game play gives a chain of states for each snake. We update
the Q values recursively, from the very last state to the first. We know what
$\max_{a_i}Q(s_1,a_i)$ is, namely the max value of the successor of $s_0$ recorded
in the chain of states. Assuming the network is already somewhat accurate, we
don’t want to change $Q(s_0,a)$ unless $\max_{a_i}Q(s_1,a_i)$ has been refined. If
$\max_{a_i}Q(s_1,a_i)$ was updated to $\max_{a_i}Q^*(s_1,a_i)$ during the
previous step, then we want the new $Q^*(s_0,a)$ to be

\begin{dmath}
  \Phi(s_0,a,s_1)\times \max_{a_i}Q^*(s_1,a_i)+\sum\limits_{s_i\neq
    s_1}\Phi(s_0,a,s_i)\max_{a_j}Q(s_i,a_j) = \Phi(s_0,a,s_1)\times
  \max_{a_i}Q^*(s_1,a_i)+\sum\limits_{s_i\neq
    s_1}\Phi(s_0,a,s_i)\max_{a_j}Q(s_i,a_j) - \Phi(s_o,a,s_1)\times
  \max_{a_i}Q(s_1,a) = \Phi(s_0,a,s_1)\times
  [\max_{a_i}Q^*(s_1,a_i)-\max_{a_i}Q^*(s_1,a_i)]+\sum\limits_{s_i\neq
    s_1}\Phi(s_0,a,s_i)\max_{a_j}Q(s_i,a_j) = \Phi(s_0,a,s_1)\times
  [\max_{a_i}Q^*(s_1,a_i)-\max_{a_i}Q^*(s_1,a_i)]+Q(s_0,a)
\end{dmath}

% \begin{equation}
%   \Phi(s_0,a,s_1)\times \max_{a_i}Q^*(s_1,a_i)+\sum\limits_{s_i\neq s_1}\Phi(s_0,a,s_i)\max_{a_j}Q(s_i,a_j)
% \end{equation}

% \begin{equation}
%   = \Phi(s_0,a,s_1)\times \max_{a_i}Q^*(s_1,a_i)+\sum\limits_{s_i\neq s_1}\Phi(s_0,a,s_i)\max_{a_j}Q(s_i,a_j)
% \end{equation}

% \begin{equation}
%   - \Phi(s_o,a,s_1)\times \max_{a_i}Q(s_1,a)
% \end{equation}

% \begin{equation}
%   = \Phi(s_0,a,s_1)\times [\max_{a_i}Q^*(s_1,a_i)-\max_{a_i}Q^*(s_1,a_i)]+\sum\limits_{s_i\neq s_1}\Phi(s_0,a,s_i)\max_{a_j}Q(s_i,a_j)
% \end{equation}

% \begin{equation}
%   = \Phi(s_0,a,s_1)\times [\max_{a_i}Q^*(s_1,a_i)-\max_{a_i}Q^*(s_1,a_i)]+Q(s_0,a)
% \end{equation}

We can see that if the max of its successor was unchanged then the new value
$Q^*(s_0,a)$ just remains unchanged, too. There is, however, a flaw in this
updating method. It has no guarantee on that the new Q value staying in the
range [0, 1], which is the practical range for win rate, and also the range for
the sigmoid function. In practice, if the network is already somewhat good, a
bad value should almost never happen, but we are still supposed to set the bound
manually. If it does happen, say the new $Q^*(s_0,a)\geq 1$, then this one is
guaranteed to be the new maximum since all other Q values is computed by sigmoid
and thus less than 1. Then, it very likely will shoot up its parent state’s Q
value over 1 and recursively messes up the whole chain of states. Thus, we
assign the following bounded value to $Q^*(s_0,a)$.

\begin{equation}
  min(max(\Phi(s_0,a,s_1)\times [\max_{a_i}Q^*(s_1,a_i)-\max_{a_i}Q(s_1,a_i)]+Q(s_0,a),0),1)
\end{equation}

To complete the final piece of our estimation of Q values, we need to know
$\Phi$, the transition probability. If we get the actions of every snake, then
the transition is almost deterministic, except the food spawns randomly, but
that is a minor factor. Obviously, since the game is synchronous, the
probability of any snake taking an action is independent from others.
$\Phi(s_0,a,s_1)$ is almost equal to the product of some entry of each policy
calculated by other snakes. Let the transition function
$\Gamma(s,\overrightarrow{A})$ computes the successor state of $s$ given
$\overrightarrow{A}$, a vector of actions of snakes (omitting the randomness of
food).

\begin{equation}
  \Phi(s_o,a,\Gamma(s_o,\langle a, a_1, \ldots , a_k \rangle )) \approx \Pi^k_{i=1}\pi_n{(\langle Q(s_i,0),Q(s-I,1),Q(s_i,2) \rangle)}_{a_i}
\end{equation}

Here $\pi_n$ is the softmax-like function, $s_i$ is the ith snake’s view of
state $s_0$ (not including you since you explicitly made action a), and $a_i
  \in {0,1,2}$ is the action it chose. There is one exception, if a snake made
an action and died right after, then the assigned value is simply 0, since it
is usually not relevant to what last moves other snakes chose, unless it died
due to head on head collision.

Since the training data is massive, a good trick is to obtain a sample from the
training data. There are various ways to sample the data, but one thing to
realize is that late game might be far more important than the early game. The
sampling method used in the training algorithm makes an exponentially growing
gap between samples counting from the last move to the first move. Another trick
is to use batch normalization between layers in the network, because
reinforcement learning tends to change the input distribution, it introduces
instability to the learning, and batch normalization can address that problem.
Another problem we encountered during training is that after some number of
iterations, the snakes realize that if it run into something they die, so they
come up with the so called “chicken snake” strategy, where one snake just
circling around and does nothing. In other words, it stucks at a local optimum.
To avoid this, we let the snake learn in a modified environment where every turn
the health is reduced by 9, so it will quickly die from starvation and force it
to go look for food. After that, we put it into the original environment. We
call that “learn to walk before you run”. Be careful with the choice of the
decrement value. We still want it to realize that once the health gets to 1, it
will die next turn, so the decrement must be a factor of 99. If we prime
factorize 99, we can see the only options are 1, 3, 9, 11, 33, 99.

After training, the new trained network is to be compared with the old network.
A number of games are played between them (can use different competing rules,
1v3 or 2v2, etc.). If the new network’s win rate surpasses a threshold, it will
replace the old one. A good way to select the threshold is to use a binomial
distribution to figure out a value such that a worse network (say a win rate
less than 50\% in 2v2 games) has a negligible chance of surpassing it.


\section{Experiments and Discussion}

With limited computational power, we tried the algorithm on a small CNN. It
consists of 6 convolutional hidden layers, with 32, 32, 64, 64, 128, 128 filters
of size 3x3, respectively. It used “selu” as activation function
\cite{SeluPaper}, and batch normalization layers are added between the weights
and activation functions. After running for 154 iterations, we obtained the
following results.

\subsubsection*{Acknowledgments}

We'd like to thank our professor Nishant Mehta for his quick responses to any
questions we had and providing all the help needed to complete the project.

\newpage

% \small

% \bibliographystyle{ieeetr}
\bibliographystyle{unsrt}
\bibliography{bibliography-project}

% \printbibliography

\end{document}
